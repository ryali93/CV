%%%%%%%%%%%%%%%%%
% \documentclass[10pt,a4paper,academicons]{altacv}

%% Use the "normalphoto" option if you want a normal photo instead of cropped to a circle
\documentclass[10pt,a4paper,normalphoto]{altacv}

%\documentclass[10pt,a4paper,ragged2e]{altacv}

%% AltaCV uses the fontawesome and academicon fonts
%% and packages.
%% See texdoc.net/pkg/fontawecome and http://texdoc.net/pkg/academicons for full list of symbols. You MUST compile with XeLaTeX or LuaLaTeX if you want to use academicons.

% Change the page layout if you need to
\geometry{left=1cm,right=9cm,marginparwidth=6.8cm,marginparsep=1.2cm,top=1.25cm,bottom=1.25cm}

% Change the font if you want to, depending on whether
% you're using pdflatex or xelatex/lualatex
\ifxetexorluatex
  % If using xelatex or lualatex:
  \setmainfont{Carlito}
\else
  % If using pdflatex:
  \usepackage[utf8]{inputenc}
  \usepackage[T1]{fontenc}
  \usepackage[default]{lato}
\fi

% Change the colours if you want to
\definecolor{VividPurple}{HTML}{3E0097}
\definecolor{BlueFB}{HTML}{3B5998}
\definecolor{SlateGrey}{HTML}{2E2E2E}
\definecolor{LightGrey}{HTML}{666666}
\colorlet{heading}{BlueFB}
\colorlet{accent}{BlueFB}
\colorlet{emphasis}{SlateGrey}
\colorlet{body}{LightGrey}

% Change the bullets for itemize and rating marker
% for \cvskill if you want to
\renewcommand{\itemmarker}{{\small\textbullet}}
\renewcommand{\ratingmarker}{\faCircle}

%% sample.bib contains your publications
\addbibresource{sample.bib}

\begin{document}
\name{Roy Yali Samaniego}
\tagline{Ingeniero Geógrafo, Analista de datos espaciales \& Desarrollador GIS}
% Cropped to square from https://en.wikipedia.org/wiki/Marissa_Mayer#/media/File:Marissa_Mayer_May_2014_(cropped).jpg, CC-BY 2.0
\photo{2.5cm}{Roy}
\personalinfo{%
  % Not all of these are required!
  % You can add your own with \printinfo{symbol}{detail}
  \email{ryali93@gmail.com}
    \phone{953-527-306}
  \location{Lima, PERÚ}
  
   \linkedin{linkedin.com/in/ryali93}
    \twitter{@ryali93}
    
    web: https://ryali93.github.io/blog
    \github{github.com/ryali93} % I'm just making this up though.
%   \orcid{orcid.org/0000-0000-0000-0000} % Obviously making this up too. If you want to use this field (and also other academicons symbols), add "academicons" option to \documentclass{altacv}
}

%% Make the header extend all the way to the right, if you want.
\begin{fullwidth}
\makecvheader
\end{fullwidth}

%% Depending on your tastes, you may want to make fonts of itemize environments slightly smaller
\AtBeginEnvironment{itemize}{\small}

%% Provide the file name containing the sidebar contents as an optional parameter to \cvsection.
%% You can always just use \marginpar{...} if you do
%% not need to align the top of the contents to any
%% \cvsection title in the "main" bar.
\cvsection[page1sidebar]{Experiencia}

\cvevent{Analista GIS}{INGEMMET}{Mar 2018 -- Ahora}{Lima, Perú}
\begin{itemize}
\item Desarrollo de herramientas GIS (addins, aplicaciones, widgets) en el marco del GEOCATMIN y para las diferentes oficinas geocientíficas del INGEMMET.
\item Modelado de datos, diseño y desarollo de aplicaciones móviles para la recopilación de datos espaciales.
\item Manejo de herramientas GIS y bases de datos en software libre y comercial, así como manejo de lenguajes de programación Python, R, C#, Js, Oracle.
\end{itemize}
\divider

\cvevent{Asistente de Hidrología}{SENAMHI}{Nov 2017 -- Nov 2018}{Lima, Perú}
\begin{itemize}
\item Propuesta metodológica de delimitación de cabeceras de cuenca a nivel nacional con herramientas de programación y GIS. 
\item Desarrollo de herramientas GIS y análisis hidrológico para la ubicación óptima de estaciones hidrométricas.
\end{itemize}
\divider

\cvevent{Programador GIS}{INEI}{Nov 2016 -- Oct 2017}{Lima, Perú}
\begin{itemize}
\item Encargado de la segmentación rural y actualización cartográfica a nivel nacional en el marco del proyecto de Censo Nacional de Población y Vivienda. 
\item Manejo de herramientas GIS, grandes bases de datos y lenguajes de programación (Python, SQL Server) para la designación de empadronamiento rural.
\end{itemize}
\divider

\cvevent{Asistente de Hidrología y GIS}{ANA}{Nov 2015 -- Ago 2016}{Lima, Perú}
\begin{itemize}
\item Participación en proyectos hidráulicos, afianzamiento hídrico a nivel nacional aplicando herramientas de teledetección y SIG. 
\item Sistema de potencial de erosión en las cuencas del Pacífico, desarrollado en el entorno Shiny del IDE RStudio. Permitiendo la estimación del volumen muerto por erosión, ideal para la gestión del recurso hídrico.
\end{itemize}
\divider

\cvevent{Asistente de Hidrología}{CIDHMA Ingenieros}{Jun 2015 -- Oct 2015}{Lima, Perú}

\begin{itemize}
\item Asistencia en proyectos hidráulicos, elaboración de informes de inundaciones, análisis de máximas avenidas, series de tiempo para la disponibilidad hídrica, entre otros.
\item Levantamiento topográfico y batimétrico del cauce del río Rimac en el distrito de Chosica.
\end{itemize}
\divider

% \cvevent{Product Engineer}{Google}{23 June 1999 -- 2001}{Palo Alto, CA}

% \begin{itemize}
% \item Joined the company as employe \#20 and female employee \#1
% \item Developed targeted advertisement in order to use user's search queries and show them related ads
% \end{itemize}
%\cvsection{A Day of My Life}

% Adapted from @Jake's answer from http://tex.stackexchange.com/a/82729/226
% \wheelchart{outer radius}{inner radius}{
% comma-separated list of value/text width/color/detail}
% Some ad-hoc tweaking to adjust the labels so that they don't overlap
%\wheelchart{1.5cm}{0.5cm}{%
%  25/10em/accent!30/\footnotesize\\[1ex]Sleeping \& dreaming about work,
%  20/9em/accent!60/\footnotesize\\[1ex]Spending time with family,
%  5/13em/accent!10/\footnotesize\\[1ex]Hobbies,
%  20/15em/accent!40/\footnotesize\\[1ex]Working with stakeholders to define smooth business continuity,
%  30/10em/accent/\footnotesize\\[1ex]Overseeing provision of end-user services\, including help desk and technical support services,
%  5/8em/accent!20/\footnotesize\\[1ex]Planning out task for upcoming day
%}

\clearpage

% \cvsection[page2sidebar]{Publications}
% 
% \nocite{*}
% 
% \printbibliography[heading=pubtype,title={\printinfo{\faBook}{Books}},type=book]
% 
% \divider
% 
% \printbibliography[heading=pubtype,title={\printinfo{\faFileTextO}{Journal Articles}}, type=article]
% 
% \divider
% 
% \printbibliography[heading=pubtype,title={\printinfo{\faGroup}{Conference Proceedings}},type=inproceedings]
% 
%% If the NEXT page doesn't start with a \cvsection but you'd
%% still like to add a sidebar, then use this command on THIS
%% page to add it. The optional argument lets you pull up the
%% sidebar a bit so that it looks aligned with the top of the
%% main column.
% \addnextpagesidebar[-1ex]{page3sidebar}


\end{document}
